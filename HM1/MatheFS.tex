% Mathe Formelsammlung für HM1 SoSe 2011
% 2 Seiten

% Dokumenteinstellungen
% ======================================================================	

% Dokumentklasse (Schriftgröße 6, DIN A4, Artikel)
\documentclass[6pt,a4paper]{scrartcl}

% Pakete laden
\usepackage[utf8]{inputenc}		% Zeichenkodierung: UTF-8 (für Umlaute)   
\usepackage[german]{babel}		% Deutsche Sprache
\usepackage{multicol}			% Spaltenpaket
\usepackage{amsmath}
\usepackage{amssymb}
\usepackage{esint}				% erweiterte Integralsymbole
\usepackage{multicol}			% ermöglicht Seitenspalten  
\usepackage{wasysym}			% Blitz
\usepackage{graphicx}
      
% Seitenlayout und Ränder:
\usepackage{geometry}
\geometry{a4paper,landscape, left=6mm,right=6mm, top=0mm, bottom=3mm,includeheadfoot} 

%Kopf- und Fußzeile
\usepackage{fancyhdr}
\pagestyle{fancy}
\fancyhf{}

   \fancyfoot[C]{von Emanuel Regnath (Emanuel.Regnath@tum.de) und Martin Zellner (Martin.Zellner@mytum.de) - SoSe EI 2011}
   \renewcommand{\headrulewidth}{0.0pt} %obere Linie ausblenden
   \renewcommand{\footrulewidth}{0.1pt} %obere Linie ausblenden

   \fancyfoot[R]{Stand: \today \qquad \thepage}
   \fancyfoot[L]{Evtl. Fehler bitte sofort melden!!!!einself!!}
	
% Schriftart SANS für bessere Lesbarkeit bei kleiner Schrift
\renewcommand{\familydefault}{\sfdefault} 


% Custom Commands
\renewcommand{\thesubsection}{\arabic{subsection}}
\newcommand{\me}[1]{\ensuremath{\left\{#1\right\}}}
\newcommand{\dme}[2]{\ensuremath{\left\{#1\,\vert\,#2 \right\}}}
\newcommand{\abs}[1]{\ensuremath{\left\vert#1\right\vert}}
\newcommand{\un}[1]{\; \unit{#1} }
\newcommand{\unf}[2]{\;\left[ \unitfrac{#1}{#2} \right]}
\newcommand{\norm}[2][\relax]{\ifx#1\relax \ensuremath{\left\Vert#2\right\Vert}\else \ensuremath{\left\Vert#2\right\Vert_{#1}}\fi}
\newcommand{\enbrace}[1]{\ensuremath{\left(#1\right)}}
\newcommand{\nira}[1]{\ensuremath{\overset{n \rightarrow \infty}{\longrightarrow}}}
\newcommand{\os}[2]{\ensuremath{\overset{#1}{#2}}}
\makeatletter
\newcommand{\Ra}[0]{\ensuremath{\Rightarrow}}
\newcommand{\ra}[0]{\ensuremath{\rightarrow}}
\newcommand{\gk}[1]{\ensuremath{\left\lfloor#1\right\rfloor}}
\newcommand{\sprod}[2]{\ensuremath{%
  \setbox0=\hbox{\ensuremath{#2}}
  \dimen@\ht0
  \advance\dimen@ by \dp0
  \left\langle #1\rule[-\dp0]{0pt}{\dimen@},#2\right\rangle}}

% Dokumentbeginn
% ======================================================================
\begin{document}
%\section{}
% ----------------------------------------------------------------------

% Aufteilung in Spalten
\begin{multicols}{4}
      
\subsection{Allgemeines} % (fold)
\label{sub:allgemeines}


	\begin{tabular*}{\columnwidth}{@{\extracolsep\fill}ll@{}} 
	Dreiecksungleichung: &$\big|\! \abs{x}- \abs{y}\!\big| \le \abs{x \pm y} \le \abs{x} + \abs{y}$\\
	Cauchy-Schwarz-Ungleichung: & $\left| \vec x^\top \cdot \vec y \right| \le \| \vec x\| \cdot \| \vec y\|$ \\
	Bernoulli-Ungleichung: & $(1+x)^n \ge 1+nx$\\ \hline
	Aritmetrische Summenformel &  $\sum \limits_{k=1}^{n} k = \frac{n (n+1)}{2} $ \\
	Geometrische Summenformel &  $ \sum \limits_{k=0}^{n} q^k = \frac{1 - q^{n+1}}{1-q}$ \\
	Binomialkoeffizient & $\binom nk = \binom n{n-k} = \frac{n!}{k! \cdot (n-k)!}$\\
	 & $\binom{n}{0} = \binom{n}{n} = 1$
	\end{tabular*} 

	\subsubsection*{Wichtige Zahlen:}
	\begin{tabular}{@{}llll}
		$\pi \approx 3,14159$ & $e \approx 2,71828$ & $\sqrt{2} \approx 1,414$ & $\sqrt{3} \approx 1,732$ \\
	\end{tabular}


\paragraph{Fakultäten} % (fold)
\label{par:fakultaeten}
$n! = 1 \cdot 2 \cdot 3 \cdot \ldots \cdot n$ \qquad  $0! = 1! = 1$ \\
		

% paragraph fakultäten (end)
% subsubsection subsection_name (end)
% subsection allgemeines (end)


\subsection{Mengen}
% ----------------------------------------------------------------------

Eine Zusammenfassung wohlunterschiedener Elemente zu einer Menge\\
explizite Angabe: $A=\{1;2;3\}$\\
Angabe durch Eigenschaft: $A=\{n\in\mathbb N\ \vert\ 0<n<4\}$\\
\subsubsection{Für alle Mengen A,B,C gilt:}
\begin{enumerate}\itemsep-1pt
\item $\emptyset \subseteq B $
\item $A \setminus (B \cup C) = (A \setminus B) \cap (A \setminus C)$
\item $(A \cap B) \cap C = A \cap (B \cap C)$\\
	$(A \cup B) \cup C = A \cup (B \cup C)$
\item $A \cap (B \cup C) = (A \cap B) \cup (A \cap C) \\
	A \cup (B \cap C) = (A \cup B) \cap (A \cup C)$
\end{enumerate}


$\mathbb Q=\{\frac{p}{q}\ \vert\ p\in\mathbb Z; q\in\mathbb N\}$\\
\\
Jede rationale Zahl $\frac m n \in \mathbb Q$ hat ein Dezimaldarstellung.\\
$0,25\overline{54} =: a \rightarrow 10000a - 100a = 2554 -25 \Rightarrow a(9900) = 2529 \qquad \Rightarrow a = \frac{2529}{9900} = \frac{281}{1100}$

\subsection{Vollständige Induktion}
Behauptung: $f(n)=g(n)$ für $n_0 \le n \in \mathbb N$\\ 
IA: $n=n_0$: \quad Zeige $f(n_0)=g(n_0)=$wahr.\\
IV: Behauptung gilt für ein beliebiges $n\in\mathbb N$ \quad (Sei $f(n)=$wahr)\\
IS: $n \rightarrow n+1$: \quad Zeige $f(n+1)=\underset{=wahr}{f(n)}  \dotsc=g(n+1)$

\subsection{Komplexe Zahlen}
% ----------------------------------------------------------------------
Eine komplexe Zahl $z=a+b\mathbf{i},\ z\in \mathbb C a,b \in \mathbb R$ besteht aus einem Realteil $\Re(z)=a$ und einem Imaginärteil $\Im(z)=b$, wobei $\mathbf{i}=\sqrt{-1}$ die immaginären Einheit ist.
Es gilt: \quad $i^2 = -1$ \quad $i^4 = 1$
\subsubsection{Kartesische Koordinaten}
Rechenregeln:\\
$z_1+z_2=(a_1+a_2)+(b_1+b_2)\mathbf{i}$\\
$z_1\cdot z_2=(a_1\cdot a_2-b_1\cdot b_2)+(a_1\cdot b_2+a_2\cdot b_1)\mathbf{i}$\\
\\
Konjugiertes Element von $z=a+b\mathbf{i}$:\\
$\overline{z}=a-b\mathbf{i}$\qquad \qquad \qquad \qquad \qquad \qquad \qquad \qquad $e^{\overline{ix}} = e^{-ix}$  \\
$z\overline{z}=|z|^2=a^2+b^2$\\
\\
Inverses Element:\\
$z^{-1}=\frac{\overline z}{\overline z z}=\frac{\overline z}{a^2+b^2}=\frac{a}{a^2+b^2} - \frac{b}{a^2+b^2}\mathbf{i}$


\subsubsection{Polarkoordinaten}
$z=a+b\mathbf{i}\ne0$\ in Polarkoordinaten:\\
$z=r (\cos(\varphi)+\mathbf{i}\sin(\varphi))=r\cdot e^{\varphi \mathbf{i}}$\\
$r=|z|=\sqrt{a^2+b^2}\quad\varphi=\arg(z)=\begin{cases}+\arccos \left( \frac{a}{r}\right),  & b\ge0   \\  -\arccos \left( \frac{a}{r}\right), & b<0\end{cases}$

\begin{description}\itemsep0pt
\item[Multiplikation:] $z_1\cdot z_2=r_1 * r_2 ( \cos ( \varphi_1 + \varphi_2) + \mathbf{i} \sin (\varphi_1 + \varphi_2))$
\item[Division:] $\frac{z_1}{z_2}=\frac{r_1}{r_2} ( \cos ( \varphi_1 - \varphi_2) + \mathbf{i} \sin (\varphi_1 - \varphi_2))$
\item[n-te Potenz:] $z^n=r^n\cdot e^{n\varphi \mathbf{i}}= r^n (\cos (n \varphi) + \mathbf{i} \sin (n \varphi))$
\item[n-te Wurzel:] $\sqrt[n]{z}= z_k = \sqrt[n]{r} \left(\cos \left(\frac{\varphi + 2k\pi}{n}\right) + \mathbf{i} \sin \left(\frac{\varphi + 2k\pi}{n}\right)\right) \\ k =0,1, \ldots, n-1$
\item[Logarithmus:] $\ln(z)=\ln(r) + \mathbf{i}(\varphi + 2k\pi)$ \quad (Nicht eindeutig!)
\end{description}
Anmerkung: Addition in kartesische Koordinaten umrechnen(leichter)!

\subsection{Funktionen}
Eine Funktion $f$ ist eine Abbildung, die jedem Element $x$ einer Definitionsmenge $D$ genau ein Element $y$ einer Wertemenge $W$ zuordnet.\\
$f:D\rightarrow W,\ x \mapsto f(x):=y$\\
\\
\textbf{Injektiv}: $f(x_1)=f(x_2) \Rightarrow x_1=x_2$\\
\textbf{Surjektiv}: $\forall y\in W \exists x\in D:f(x)=y$\\ \quad (Alle Werte aus $W$ werden angenommen.)\\
\textbf{Bijektiv}: $f$ ist injektiv und surjektiv $\Rightarrow$ $f$ umkehrbar.

\subsubsection{Symmetrie einer Funktion $f$}
\textbf{Achsensymmetrie}(gerade Funktion): $f(-x)=f(x)$\\
\textbf{Punktsymmetrie}(ungerade Funktion): $f(-x)=-f(x)$\\
\\
Regeln für gerade Funktion $g$ und ungerade Funktion $u$:\\
$g_1 \pm g_2 = g_3$ \qquad $u_1 \pm u_2 = u_3$\\
$g_1 \cdot g_2=g_3$ \qquad $u_1 \cdot u_2 = g_3$ \qquad $u_1 \cdot g_1=u_3$

\subsubsection{Extrema, Monotonie und Krümmung von $f$}
$f'(x_0)\overset{!}{=}0 \quad \begin{cases}f''(x_0)<0 \ \rightarrow \ \text{Maximum (lokal)} \\ f''(x_0)>0 \ \rightarrow \ \text{Minimum (lokal)}\end{cases} $\\
$f'(x) \underset{(>)}{^{\ge}} / \underset{(<)}{^{\le}} 0 \ \rightarrow$ \ $f$ (streng) Monoton steigend/fallend. $x\in[a,b]$\\
$f''(x) \underset{(>)}{^{\ge}} / \underset{(<)}{^{\le}} 0 \ \rightarrow$ \ $f$ (strikt) konvex/konkav. $x\in[a,b]$\\

\subsubsection{Asymptoten von $f$}
Horizontal: $c=\lim\limits_{x\ra \pm \infty} f(x)$\\
Vertikal: $\exists \text{ Nullstelle } a \text{ des Nenners }: \lim\limits_{x \rightarrow a^{\pm}} f(x) = \pm \infty$\\
Polynomasymptote $P(x)$: $f(x):=\frac{A(x)}{Q(x)}=P(x)+ \underset{\ra 0}{\frac{B(x)}{Q(x)}}$


\subsubsection{Wichtige Sätze für \underline{stetige} Fkt. $f: [a,b] \rightarrow \mathbb R, f \mapsto f(x)$ }
\textbf{Zwischenwertsatz:} $\forall y \in [f(a),f(b)]\ \exists x\in [a,b]:f(x)=y$\\
\textbf{Mittelwertsatz:} Falls $f$ diffbar, dann $\exists x_0:f'(x_0)=\frac{f(b)-f(a)}{b-a}$\\
\textbf{Satz von Rolle:} Falls $f(a)=f(b)$, dann $\exists x_0: f' (x_0) = 0$\\
\textbf{Regel von L'Hospital}: (Falls $\exists$ ein Grenzwert) \\ $\lim\limits_{x \rightarrow a} \frac{f(x)}{g(x)} \rightarrow \left[ \frac{0}{0} \right] / \left[ \frac{\infty}{\infty} \right] = \lim\limits_{x \rightarrow a} \frac{f'(x)}{g'(x)}$

% ----------------------------------------------------------------------
\subsubsection{Polynome $P(x)\in\mathbb R[x]_n$}
$P(x)=\sum_{i=0}^n a_ix^i=a_n x^n+a_{n-1} x^{n-1}+\dotsc+a_1x+a_0$ \\
Lösungen für $ax^2+bx+c=0$ \\
\begin{tabular}{l|l}
Mitternachtsformel:  &  Satz von Vieta:\\
$x_{1/2}=\frac{-b\pm\sqrt{b^2-4ac}}{2a}$  \quad & \quad   $x_1 + x_2 = - \frac{b}{a} \qquad x_1 x_2 = \frac{c}{a}$
\end{tabular}

\subsubsection{Trigonometrische Funktionen}
$f(t)=A\cdot \cos(\omega t + \varphi_0)=A\cdot \sin(\omega t + \frac{\pi}{2}+ \varphi_0)$
\begin{eqnarray*}
	\sin (-x) = -\sin (x)  \quad & \quad \cos (-x) = \cos (x) \\
	\sin^2 x + \cos^2 x = 1  \quad & \quad \tan x = \frac{\sin x}{\cos x}
\end{eqnarray*}
$e^{ix}=cos(x)+i\,sin(x)$, $e^{-ix}=sin(x)-i\,cos(x)$

\paragraph{Additionstheoreme} % (fold)
\label{par:additionstheoreme}
 \begin{eqnarray*}
 	\cos (x + y) = \cos x \cos y - \sin x \sin y \\
	\cos \enbrace{x - \frac{\pi}{2}} = \sin x \qquad \quad \sin \enbrace{x + \frac{\pi}{2}} = \cos x \\
    \sin \enbrace{x + y} = \sin x \cos y + \cos x \sin y \\
	\sin 2x = 2 \sin x \cos x        \\
	\cos 2x = \cos^2 x - \sin^2 x = 2\cos^2 x - 1\\
 \end{eqnarray*}
% paragraph additionstheoreme (end)

$\begin{array}{c|c|c|c|c|c|c|c|c}
x & 0 & \pi / 6 & \pi / 4 & \pi / 3 & \pi / 2 & \pi & \frac{3}{2}\pi & 2 \pi \\ \hline
\sin & 0 & \frac{1}{2} & \frac{1}{\sqrt{2}} & \frac{\sqrt 3}{2} & 1 & 0 & -1 & 0 \\
\cos & 1 & \frac{\sqrt 3}{2} & \frac{1}{\sqrt 2} & \frac{1}{2} & 0 & -1 & 0 & 1 \\     
\tan & 0 & \frac{\sqrt{3}}{3}&	1				 &	\sqrt{3} & \lightning & 0 & \lightning & 0\\
\end{array}$


\subsection{Matrizen}
% ----------------------------------------------------------------------
Eine Matrix ist eine Tabelle aus mathematischen Objekten.
Die Matrix $A=(a_{ij}) \in \mathbb K^{m\times n}$ hat $m$ Zeilen mit Index $i$ und $n$ Spalten mit Index $j$

\subsubsection{Allgemeine Rechenregeln}
\textbf{Merke:} Zeile vor Spalte! (Multiplikation, Indexreihenfolge, etc...)\\

\begin{tabular}{ll}	
	1)  $A+0=A$ & 2)  $1 \cdot A=A$ \\
	3)  $A+B=B+A$ & 4) $A \cdot B \ne B \cdot A$ (im allg.) \\
	5)  $(A+B)+C=A+(B+C)$ & 6) $\lambda (A+B) = \lambda A + \lambda B$\\ 
\end{tabular}
Multiplikation von $A\in \mathbb K^{m\times r}$ und $B\in \mathbb K^{r\times n}$: $AB\in\mathbb K^{m\times n}$

\subsubsection{Transponieren}
Falls $A=(a_{ij})\ \in \mathbb K^{m\times n}$ gilt: $A^\top=(a_{ji})\ \in \mathbb K^{n\times m}$\\
Regeln:\\
$(A+B)^\top=A^\top+B^\top$\qquad $(A\cdot B)^\top=B^\top\cdot A^\top$\qquad \\ $(\lambda A)^\top=\lambda A^\top$ \qquad $(A^\top)^\top=A$\\
\\
$A\in \mathbb K^{n\times n}$ ist symmetrisch, falls $A=A^\top$\qquad ($\Rightarrow$ diagbar)\\
$A\in \mathbb K^{n\times n}$ ist schiefsymmetrisch, falls $A=-A^\top$\\
$A\in \mathbb K^{n\times n}$ ist orthogonal(Spaltenvektoren=OGB), falls:\\
\qquad\ $AA^\top=E_n$\qquad $A^\top=A^{-1}$\qquad $\det A=\pm 1$\\
$A\in \mathbb C^{n\times n}$ ist hermitesch, falls $A=\overline{A}^\top$  \quad (kmplx. konj. u. transp.)


\subsubsection{Inverse Matrix $A^{-1}\in \mathbb K^{n\times n}$}
für die inverse Matrix $A^{-1}$ von $A\in \mathbb K^{n\times n}$ gilt: $A^{-1}A=E_n$\\
$(A^{-1})^{-1}=A$ \qquad $(AB)^{-1}=B^{-1}A^{-1}$ \qquad $(A^\top)^{-1}=(A^{-1})^\top$\\
\\
$A\ \in \mathbb R^{n\times n}$ ist invertierbar, falls: $\det (A) \ne 0 \quad \lor \quad rg(A)=n$\\
\\
Berechnen von $A^{-1}$ nach Gauß:\\
$AA^{-1}=E_n\quad\Rightarrow\quad (A|E_n)\overset{EZF}{\longrightarrow}(E_n|A^{-1})$\\

\subsubsection{Elementare Zeilen/Spaltenumformungen(EZF/ESF)}
$A \in \mathbb K^{m\times n}$ hat $m$ Zeilen $z_i\in \mathbb K^n$ und $n$ Spalten $s_j\in \mathbb K^m$
\begin{itemize}\itemsep0pt
\item \textbf{Addition} ($\lambda\ne 0$):\quad $\lambda_1 z_1 + \lambda_2  z_2$ \quad / \quad $\lambda_1  s_1 + \lambda_2 s_2$
\item Vertauschen von Zeilen/Spalten
\item Multiplikation mit $\lambda\ne 0$: \quad $\lambda \cdot z$ \quad  / \quad  $\lambda \cdot s$
\end{itemize}

\subsubsection{Rang einer Matrix $A$}
$A\in \mathbb K^{m\times n}$ mit $r$ lin. unabhängige Zeilen und $l$ "Nullzeilen":\\
Rang von $A$: $\mathrm{rg}(A)=m-l=r$\\  
Vorgehensweise: \\
\textbf{Zeilenrang (A):} Bringe $A$ auf ZSF $\Ra$ Zeilenrang$(A) = rg(A)$\\     
\textbf{Zeilenraum (A):}  $Z_A = \text{ Zeilen ungleich } 0$            \\
\textbf{Spaltenrang:} Bringe Matrix auf Spaltenstufenform        \\
\textbf{Kern:  }   $\ker(A) = \dme{x \in \mathbb R^n}{Ax= 0}$ \qquad $\mathrm{dim}(\ker(A))=n-r$ \\
\textbf{Bild: } $A^T \Ra EZF \Ra $ Zeilen $(\not= 0)$ bilden die Basis vom Bild. Die (lin. unabhängigen) Spalten von $A$ bilden eine Basis vom Bild.
\subsubsection{Lineares Gleichungssystem LGS}
Das LGS $Ax=b$ kurz $(A|b)$ mit $A\in \mathbb K^{m\times n}$, $x\in \mathbb K^n$, $b\in \mathbb K^m$ hat $m$ Gleichungen und $n$ Unbekannte.\\
\\
\textbf{Lösbarkeitskriterium:}\\
Ein LGS $(A|b)$ ist genau dann lösbar, wenn: $\mathrm{rg}(A)=\mathrm{rg}(A|b))$\\
Die Lösung des LGS $(A|b)$ hat $\dim{\ker A} = n-\mathrm{rg}(A)$ frei wählbare Parameter.\\
\\
Das homogene LGS: $(A|0)$ hat stets die triviale Lösung $0$\\
Das LGS hat eine Lsg. wenn $\det A \not= 0$ \quad $\rightarrow \exists A^{-1}$ \\
Summen und Vielfache der Lösungen von $(A|0)$ sind wieder Lösungen.

\subsubsection{Determinante von $A\in \mathbb K^{n\times n}$: $\det(A)=|A|$}

\begin{itemize}\itemsep0pt
\item $\det\begin{pmatrix}A&0\\C&D\end{pmatrix}=\det\begin{pmatrix}A&B\\0&D\end{pmatrix}=\det(A)\cdot\det(D)$
\item $\begin{vmatrix}\lambda_1&&* \\ &\ddots& \\ 0&&\lambda_n \end{vmatrix} = \lambda_1\cdot \ldots\cdot \lambda_n = \begin{vmatrix} \lambda_1&&0  \\  &\ddots& \\  *&&\lambda_n \end{vmatrix}$
\item $A=B \cdot C \quad \Rightarrow \quad |A|=|B| \cdot |C|$
\item $\det(A)=\det(A^\top)$
\item Hat $A$ zwei gleiche Zeilen/Spalten $\Rightarrow |A|=0$
\item $|A|=\sum\limits_{i=1}^n (-1)^{i+j} \cdot a_{ij} \cdot |A_{ij}|$ \qquad Entwcklng. n. $i$ter Zeile.
\item $\det(\lambda A)=\lambda^n \det(A)$
\item Ist $A$ invertierbar, so gilt: $\det(A^{-1})=(\det(A))^{-1}$
\item Vertauschen von Zeilen/Spalten ändert Vorzeichen von $|A|$
\item $\det(AB) = \det(A) \det(B) = \det(B) \det(A) = \det(BA)$
\end{itemize}
Vereinfachung für Spezialfall $A\in \mathbb K^{2\times 2}$:\\
$A=\begin{pmatrix}a&b\\c&d\end{pmatrix}\in \mathrm{K}^{2\times 2}\Rightarrow \det(A)=|A|=ad-bc$






\subsection{Vektorräume}
% --------------------------------------------------------------
Eine nichtleere Menge V mit zwei Verknüpfungen $+$ und $\cdot$ heißt $K$-Vektorraum über dem Körper $\mathbb K$.\\
\textbf{Linear Unabhängig:} Vektoren heißen linear unabhängig, wenn aus: \\
$\lambda_1 \vec v_1 + \lambda_2 \vec v_2 + \ldots + \lambda_n \vec v_n = \vec 0$ folgt, dass $\lambda_1 = \lambda_2 = \lambda_n = 0$
\subsubsection{Skalarprodukt $\langle v,w \rangle$} 
	\begin{description}
	\item[Bilinear:] $\langle \lambda v+v',w \rangle=\lambda\cdot\langle v,w \rangle + \langle v',w \rangle$
	\item[Symmetrisch:] $\langle v,w \rangle=\langle w,v \rangle$
	\item[Positiv definit:] $\langle v,v \rangle\ge0$ 
	\end{description}  
Skalarprodukt bezüglich \textbf{symmetrischer, quadratischer} und \textbf{positiv definite} Matrix $A\in \mathbb R^{n\times n}$\\
$\langle v,w \rangle_A=v^T A w$\\
Matrix A positiv definit falls $\det (a_{11}) > 0 \land \det \left(\begin{matrix} a_11 & a_12\\ a_21 & a_22\end{matrix}\right) > 0 \land \dotsc \land \det (A)>0$   \\
\textbf{Orthogonale Projektion} $p\in U^n$ von $q\in V^m$ auf $\sum u_i$:
\begin{eqnarray*}
   p=\sum_{i=1}^n \sprod{q}{\frac{u_i}{\abs{u_i}}}\frac{u_i}{\abs{u_i}} \quad = q - p^\perp
\end{eqnarray*} 
\textbf{Winkel} \quad 	$\sprod{\vec a}{\vec b} = a \cdot b \cdot \cos \phi$ \qquad
$\phi = \arccos \enbrace{ \frac{\sprod{x}{y} }{\norm{x} \norm{y} } }$\\
\textbf{Polynome} $<p(x),q(x)>=\int\limits_{0}^{1}p(x)q(x)\,dx$

\subsubsection{Betrag von Vektoren}
$
||\vec a||=\sqrt{<\vec a,\vec a>} =\sqrt{a_1^2+a_2^2+\ldots +a_n^2}
$


\subsubsection{Orthogonalität}
\textbf{Orthonormalisierungsvefahren von $n$ Vektoren nach Gram-Schmidt:}\\
1. $b_1=\frac{v_1}{\|v_1\|}$ \qquad (Vektor mit vielen 0en oder 1en)\\
2. $b_{k+1}= \frac{b_{k+1}^{'}}{\|b_{k+1}^{'}\|}$\ \ mit \ \ $b_{k+1}^{'}=v_{k+1}-\sum_{i=1}^k \langle v_{k+1},b_i \rangle \cdot b_i$\\
\\
\textbf{Ausgleichsrechnung:}\\
Experiment: $(t_1,y_1), \hdots, (t_n,y_n)$\\
$f_1: \mathbb R \rightarrow \mathbb R, f_1(x) =1$ \qquad $f_2: \mathbb R \rightarrow \mathbb R, f_2(x) = x$
\begin{eqnarray*}
\Rightarrow A = \begin{pmatrix}f_1(t_1) & f_2(t_1)\\\vdots & \vdots\\f_1(t_n) & f_2(t_n)\end{pmatrix} \qquad v = \begin{pmatrix}y_1\\\vdots\\y_n\end{pmatrix}
\end{eqnarray*}
$A^{\top}Ax=A^{\top}v \rightarrow $ LGS lösen nach x\\
$f: \mathbb R \rightarrow \mathbb R, f(x) = x_1 f_1(x) + \hdots + x_n f_n(x)$
\\
\textbf{Orthogonale Projektion in UVR:} \quad \\
1. Normiere Basis von $U$. \\
2. $u = \sprod{b_1}{v}b_1 + \sprod{b_2}{v}b_2 \ldots \Ra u^\perp = v - u$ \\
Abstand von $v$ zu $U$: $\norm{u^{\perp}}$

\subsubsection{Vektorprodukt}
$\vec a\times\vec b=\left( \begin{matrix} a_2b_3-a_3b_2\\a_3b_1-a_1b_3\\a_1b_2-a_2b_1\end{matrix}\right)$\qquad $\vec a,\vec b\ \in \mathbb R^3$\\
\\
$\vec a\times\vec b \perp \vec a,\vec b$ \qquad $\vec a\times\vec b=0\ \Leftrightarrow\ \vec a;\vec b$\ linear abhängig.\\
$||\vec a\times\vec b||=||\vec a||\cdot||\vec b||\cdot \sin\left(\measuredangle (\vec a;\vec b)\right)\mathrel{\widehat{=}}$\ Fläche des Parallelogramms\\
Graßmann-Identität: $\vec a\times(\vec b \times \vec c)\equiv\vec b\cdot(\vec a \cdot \vec c)-\vec c\cdot(\vec a \cdot \vec b)$\\
\\
Spatprodukt:\\
$[a,b,c]:=\langle \vec a\times\vec b,\vec c\rangle=\det (\vec a, \vec b,\vec c)\mathrel{\widehat{=}}$\ Volumen des Spates.\\
$[a,b,c]>0\ \Rightarrow\ a,b,c$\ bilden Rechtssystem \\ $[a,b,c]=0\ \Rightarrow\ a,b,c$\ linear abhängig\\ \\
Orthogonale Zerlegung eine Vektors v längs a:\\
$v = v_a + v_{a^\perp} \text{ mit } v_a = \frac{\sprod{v}{a} }{\sprod{a}{a} }*a \text{ und }	 v_{a^\perp} = v - v_a	$
\subsubsection{Basis (Jeder VR besitzt eine Basis!)} % (fold)
\label{sub:basis}
 Eine Teilmenge $B$ heißt Basis, von $V$ wenn gilt:
\begin{itemize}\itemsep0pt
	\item $\left\langle B \right\rangle =V$  $B$ erzeugt $V$
	\item $B$ ist linear unabhängig
\end{itemize}                   


% subsection basis (end)
\subsubsection{Dimension} % (fold)
\label{sub:dimension}
  \begin{eqnarray*}
  	   n:= \abs{B} \in \mathbb N_0 \text{ Dimension von }V \quad & \quad \dim (V) = n
  \end{eqnarray*}   
Mehr als $n$ Vektoren sind stehts linear abhängig. \\
Für jeden UVR $U \subset V$ gilt: $\dim (U) < \dim (V)$ 
% subsection dimension (end)






\subsection{Untervektorräume}
Eine Teilmenge $U$ eines $K-$Vektorraums $V$ heißt Untervektorraum (U-VR) von $V$, falls gilt:
\begin{enumerate}\itemsep0pt
\item $U\neq \varnothing$ \qquad ($0\in U$)
\item $u+v\in U \quad \forall u,v\in U$
\item $\lambda u \in U \qquad \forall u\in U,\forall \lambda \in K$
\end{enumerate}
Wegen (3.) enthält ein UVR $U$ stets den Nullvektor $0$. Daher zeigt man (1.) meist, indem man $0\in U$ nachweist.\\
\\
\textbf{Triviale UVR}: $U=\{0\}$ mit $B = \emptyset$ \qquad $U=V$ mit $B_U=B_V$



\subsection{Folgen}
% ----------------------------------------------------------------------
Eine Folge ist eine Abbildung $a: \mathbb N_0 \rightarrow \mathbb R,\ n \rightarrow a(n) =: a_n$\\
explizite Folge: $(a_n)$ mit $a_n=a(n)$\\ rekursive Folge: $(a_n)$ mit $a_0=f_0,\  a_{n+1}=a(a_n)$\\

\subsubsection{Monotonie}
Im Wesentlichen gibt es 3 Methoden zum Nachweis der Monotonie:
\begin{enumerate}\itemsep0pt
\item $a_{n+1} - a_n \gtrless (=) 0$
\item $\frac{a_n}{a_{n+1}} \gtrless (=) 1$ \qquad $\lor$ \qquad $\frac{a_{n+1}}{a_n} \lessgtr (=) 1$
\item Vollständige Induktion
\end{enumerate}

\subsubsection{Konvergenz}
$(a_n)$ ist \emph{Konvergent} mit \emph{Grenzwert} $a$, falls: $\forall \epsilon > 0\, \exists N  \in \mathbb N_0:  \abs{a_n -a} < \epsilon  \forall n \ge N$\\
Eine Folge konvergiert gegen eine Zahl $a$:\ $(a_n) \overset{n \rightarrow \infty}{\longrightarrow} a$\\
\paragraph{Es gilt:}
\begin{itemize}\itemsep0pt
\item Der Grenzwert a einer Folge $(a_n)$ ist eindeutig.
\item Ist $(a_n)$ Konvergent, so ist $(a_n)$ beschränkt
\item Ist $(a_n)$ unbeschränkt, so ist $(a_n)$ divergent.
\item \emph{Das Monotoniekriterium}: Ist $(a_n)$ beschränkt und monoton, so konvergiert $(a_n)$
\item \emph{Das Cauchy-Kriterium:} Eine Folge $(a_n)$ konvergiert gerade dann, wenn: \\ $\forall \epsilon >0 \, \exists \,  N \in \mathbb N_0: \abs{a_n - n_m} < \epsilon \, \forall n, m \ge N$
\end{itemize}
Regeln für konvergente Folgen $(a_n) \overset{n \rightarrow \infty}{\longrightarrow} a$ und $(b_n) \overset{n \rightarrow \infty}{\longrightarrow} b$:\\
$\begin{array}{lll}
(a_n+b_n) \overset{n \rightarrow \infty}{\longrightarrow} a+b & (a_n b_n) \overset{n \rightarrow \infty}{\longrightarrow} ab & (\frac{a_n}{b_n}) \overset{n \rightarrow \infty}{\longrightarrow} \frac{a}{b}\\
(\lambda a_n) \overset{n \rightarrow \infty}{\longrightarrow} \lambda a & (\sqrt{a_n}) \overset{n \rightarrow \infty}{\longrightarrow} \sqrt{a} & (|a_n|) \overset{n \rightarrow \infty}{\longrightarrow} |a|
\end{array}$

\subsubsection{Wichtige Regeln}
$a_n=q^n \quad \overset{n \rightarrow \infty}{\longrightarrow} \quad \begin{cases} 0 & |q|<1 \\ 1 & q=1 \\ \pm \infty & q < -1  \\  + \infty & q > 1\end{cases}$ \\
$a_n=\frac{1}{n^k}\rightarrow 0$ \qquad $\forall k \ge 1$\\ 
$a_n=\left(1+\frac{c}{n}\right)^n \rightarrow e^c$ \qquad \qquad \qquad $2^n \ge n^2$ \quad $\forall n\ge 4$





\subsection{Reihen}
% ----------------------------------------------------------------------
\begin{eqnarray*}
	\underset{\text{Harmonische Reihe}}{\sum_{n=1}^\infty \frac{1}{n} = \infty} \qquad \qquad  \qquad \qquad \underset{\text{Geometrische Reihe}}{\sum_{n=0}^\infty q^n = \frac{1}{1-q}} \qquad |q|<1
\end{eqnarray*}

\subsubsection{Konvergenzkriterien}
$\sum^{\infty}_{n = 0} a_n$ divergiert, falls $a_n \not \rightarrow 0$ oder\\
Minorante:$\exists \sum^{\infty}_{n = 0} b_n (div) \quad \land \quad a_n \ge b_n \ \forall n\ge n_0$\\[0.6em] 
$\sum^{\infty}_{n = 0}(-1)^n a_n$ konvergiert falls $(a_n)$ monoton fallende Nullfolge\\
oder Majorante: $\exists \sum^{\infty}_{n = 0} b_n = b \quad \land \quad a_n \le b_n \ \forall n\ge n_0$\\
\\
Absolute Konvergenz($\sum^\infty_{n=0} |a_n|=a$ konvergiert), falls:\\
1. Majorante: $\exists \sum^{\infty}_{n = 0} b_n = b \quad \land \quad |a_n| \le b_n \quad \forall n\ge n_0$\\
2. Quotienten und Wurzelkriterium:
\begin{eqnarray*}
	\rho := \lim_{n \rightarrow \infty} \abs{\frac{a_{n+1}}{a_n}} \qquad \lor \qquad \rho := \lim_{n \rightarrow \infty} \sqrt[n]{\abs{a_n}}\\
	\text{Falls} 
	\begin{cases}
		\rho < 1 \Ra  ~\sum^\infty_{n=0} a_n \text{ konvergiert absolut} \\
		\rho > 1 \Ra  ~\sum^\infty_{n=0} a_n \text{ divergiert} \\
		\rho = 1 \Ra ~ \sum^\infty_{n=0} a_n \text{ keine Aussage möglich}
	\end{cases}
\end{eqnarray*}




\subsection{Potenzreihen} % (fold)
% ----------------------------------------------------------------------
\begin{equation*}
f(x)=\sum_{n=0}^\infty a_n \cdot (x-a)^n
\end{equation*}
Konvergenz:\\
$\abs{\frac{a_{n+1} (x-a)^{n+1}}{a_n (x-a)^n}} = \abs{\frac{a_{n+1}}{a_n}}\abs{x-a} \overset{n \rightarrow \infty}{\rightarrow} q \cdot \abs{x -a}$\\
Falls $\begin{cases}  \abs{x-a} < \frac{1}{q} & \text{ konvergiert absolut}\\
	\abs{x-a} > \frac{1}{q} & \text{ divergiert} \\
	\abs{x-a} = \frac{1}{q}  & \text{ keine Aussage möglich}
	\end{cases}$\\
Konvergenzradius: $R=\frac{1}{q}$\\
$R = \underset{n\rightarrow \infty}{\lim} \abs{\frac{a_n}{a_{n+1}}}=\lim\limits_{n\rightarrow \infty}\frac{1}{\sqrt[n]{\abs{a_n}}}$

\label{sub:potenzreihen}
 \begin{eqnarray*}
 	e^z = \sum_{n = 0}^{\infty} \frac{z^n}{n!}\\
	\sin (z) = \sum_{n = 0}^{\infty} (-1)^n \frac{z^{2n +1}}{(2n +1)!} = \frac{e^{iz} - e^{-iz}}{2i} \\
	\cos (z) = \sum_{n = 0}^{\infty} (-1)^n \frac{z^{2n}}{(2n)!} = \frac{e^{iz} + e^{-iz}}{2}\\
 \end{eqnarray*}
% subsection potenzreihen (end)





\subsection{Ableitung und Integral}
$f$ diffbar, falls $f$ stetig und $\underset{h\rightarrow 0}{\lim}\frac{f(x_0+h)-f(x_0)}{h}=f'(x)$ exist.
\subsubsection{Ableitungsregeln:}
Linearität: $(\lambda f + \mu g)' (x) = \lambda f'(x) + \mu g'(x)$ \quad $\forall \lambda, \mu \in \mathbb R$ \\
Produktregel: $(f \cdot g)'(x) = f'(x) g(x) + f(x) g'(x)$\\
Quotientenregel$\enbrace{\frac{\text{NAZ}-\text{ZAN}}{\text{N}^2}}$: $\enbrace{\frac{f}{g}}' (x) = \frac{g(x)f'(x) -f(x) g'(x)}{g(x)^2}$\\
Kettenregel: $\left( f(g(x)) \right)' = f'(g(x)) g'(x)$\\
Potenzreihe: $f: ] \underbrace{-R+a, a+R}_{\subseteq D}	 [ \rightarrow \mathbb R, f(x) = \sum_{n=0}^{\infty} a_n (x -a)^n$ \quad $\Rightarrow$ \quad $f'(x) = \sum_{n=0}^{\infty} n a_{n} (x-a)^{n-1}$\\
\textbf{Tangentengleichung:} $y=f(x_0)+f'(x_0)(x-x_0)$

\subsubsection{Newton-Verfahren:}
$x_{n+1}=x_n-\frac{f(x_n)}{f'(x_n)}$ mit Startwert $x_0$

\subsubsection{Integrationsmethoden:}
\begin{itemize}\itemsep0pt
\item Anstarren + Göttliche Eingebung
\item Partielle Integration: $\int uv'=uv-\int u'v$
\item Substitution: $\int f(\underbrace {g(x)}_{t}) \underbrace {g'(x)\,\mathrm dx}_{\mathrm dt}=\int f(t)\, \mathrm dt$
\item Brechstange: $t=\tan(\frac{x}{2})$ \quad $\mathrm dx = \frac{2}{1+t^2} \mathrm dt$ \\ $\sin(x) \rightarrow \frac{2t}{1+t^2}$ \qquad $\cos(x) \rightarrow \frac{1-t^2}{1+t^2}$
\end{itemize}

\subsubsection{Integrationsregeln:}
$\int_a^b f(x) \mathrm dx = F(b) - F(a)$\\
$\int\lambda f(x)+\mu g(x) \, \mathrm dx=\lambda\int f(x) \, \mathrm dx + \mu\int g(x) \, \mathrm dx$

\everymath{\displaystyle}	% Formeln ab hier groß Schreiben
\begin{math}\renewcommand{\arraystretch}{1.8}
\begin{array}{c|c|c}
F(x) & f(x) & f'(x) \\ \hline 
\frac{f'(x)}{f(x)}&ln|f(x)|&\frac{1}{f(x)}\cdot f'(x)\\
\frac{1}{q+1}x^{q+1} & x^q & qx^{q-1} \\
\frac{2\sqrt{x^3}}{3} & \sqrt{x} & \frac{1}{2\sqrt{x}}\\
x\ln(x) -x & \ln(x) & \textstyle \frac{1}{x}\\
e^x & e^x & e^x \\
\frac{a^x}{\ln(a)} & a^x & a^x \ln(a) \\
-\cos(x) & \sin(x) & \cos(x)\\
\sin(x) & \cos(x) & -\sin(x)\\
-\ln |\cos(x)| & \tan(x) & \frac{1}{\cos^2(x)} \\
\ln |\sin(x)| & \cot(x) & \frac{-1}{\sin^2(x)} \\
x\arcsin (x)+\sqrt{1-x^2} & \arcsin(x) & \frac{1}{\sqrt{1-x^2}}\\
x\arccos (x)-\sqrt{1-x^2} & \arccos(x) & -\frac{1}{\sqrt{1-x^2}}\\
x\arctan (x)-\frac{1}{2} \ln \left| 1+ x^2 \right| & \arctan (x) & \frac{1}{1+x^2} \\
\end{array}
\end{math}
\everymath{\textstyle}


\subsubsection{Rotationskörper}
Volumen: $V = \pi \int_a^b f(x)^2 \mathrm dx$\\
Oberfläche: $O = 2 \pi \int_a^b f(x) \sqrt{1 + f'(x)^2} \mathrm dx$

\subsubsection{uneigentliches Integral}
$\int\limits_{\text{ok}}^{\text{böse}} f(x) \mathrm dx = \lim\limits_{b\rightarrow \text{böse}}\ \int\limits_{\text{ok}}^b f(x) \mathrm dx$\\
\\
Majoranten-Kriterium: $|f(x)|\le g(x)$\\
$\int\limits_{1}^{\infty} \frac{1}{x^\alpha} \mathrm dx \begin{cases} \frac{1}{\alpha -1}, \quad \alpha > 1 \\ \infty, \qquad \alpha \le 1 \end{cases}$ \qquad
$\int\limits_{0}^{1} \frac{1}{x^\alpha} \mathrm dx \begin{cases} \frac{1}{\alpha -1}, \quad \alpha < 1 \\ \infty, \qquad \alpha \ge 1 \end{cases}$\\
\textbf{Cauchy-Hauptwert:} $\int\limits_{-\infty}^{\infty} f(x) \mathrm dx = \lim\limits_{b\rightarrow\infty} \int\limits_{-b}^b f(x) \mathrm dx$

\subsubsection{Laplace-Transformation von $f: [0,\infty[ \ra \mathbb R,\ s \mapsto f(s)$}
$\mathcal L \; f(s) = F(s) = \int\limits_{0}^{\infty} e^{-st} f(t)\ \mathrm dt = \lim\limits_{b \ra \infty} \int\limits_{0}^{b} e^{-st} f(t)\ \mathrm dt$

\subsubsection{Integration rationale Funktionen}
Gegeben: $\int \frac{A(x)}{Q(x)} \mathrm dx \qquad A(x),Q(x)\in \mathbb R[x]$
\begin{enumerate}\itemsep0pt
\item Falls, $\deg A(x) \ge \deg Q(x) \Ra$ Polynomdivision: \\ $\frac{A(x)}{Q(x)} = P(x) + \frac{B(x)}{Q(x)}$ mit $\deg B(x) < \deg Q(x)$
\item Zerlege $Q(x)$ in unzerlegbare Polynome
\item Partialbruchzerlegung $\frac{B(x)}{Q(x)} = \frac{\ldots}{(x - a_n)} + \ldots + \frac{\ldots}{\ldots}$
\item Integriere die Summanden mit folgenden Funktionen
\end{enumerate}

$\text{mit} ~ \lambda=x^2+px+q, ~~ \beta=4q-p^2 ~~ \text{und} ~p^2<4q$!
$\int\frac{1}{(x-a)^m}\mathrm dx \begin{cases} \ln\left|x-a\right|, & m=1\\ \frac{-1}{(m-1)(x-a)^{m-1}} &m\geq2 \end{cases}$\\
$\int\frac{1}{(\lambda)^m} \mathrm dx \begin{cases} \frac{2}{\sqrt{\beta}} \arctan\frac{2x+p}{\sqrt{\beta}}, &m=1\\ \frac{2x+p}{(m-1)(\beta)(\lambda)^{m-1}}+\frac{2(2m-3)}{(m-1)(\beta)} \int\frac{\mathrm dx}{(\lambda)^{m-1}}, &m\geq2 \end{cases}$\\
$\int\frac{Bx+C}{(\lambda)^m} \mathrm dx \begin{cases} \frac{B}{2} \ln(\lambda) + (C-\frac{Bp}{2}) \int\frac{\mathrm dx}{\lambda}, &m=1\\ \frac{-B}{2(m-1)(\lambda)^{m-1}} + (C-\frac{Bp}{2}) \int\frac{\mathrm dx}{(\lambda)^{m-1}}, &m\geq2 \end{cases}$\\
Auch wichtig: Schrödinger's Katze:\includegraphics[height=0.5cm]{./cat.jpg}	% {../../../../Latex4Uni/Latex4UniGIT/Mathe/Formelsammlung/cat.jpg}
\end{multicols}
% Ende der Spalten


% Dokumentende
% ======================================================================
\end{document}
